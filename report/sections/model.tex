To mathematically model an incompressible viscous fluid, we use the Navier-Stokes equations.

\begin{equation}
\frac{\partial}{\partial t} \mathbf{u} - \nu \Delta \mathbf{u} + \mathbf{u} \cdot \nabla \mathbf{u} + \nabla p = \mathbf{f} \text{ in } \Omega, t > 0 
\end{equation}
\begin{equation}
\nabla \cdot \mathbf{u} = 0 \text{ in } \Omega, t > 0
\end{equation}
\begin{equation}
\mathbf{u} = \mathbf{g}_D \text{ on } \Gamma_D, t > 0
\end{equation}
\begin{equation}
\nu \frac{\partial \mathbf{u}}{\partial n} - p \mathbf{n} = \mathbf{g}_N \text{ on } \Gamma_N, t > 0
\end{equation}
\begin{equation}
\mathbf{u} = \mathbf{u}_0 \text{ in } \Omega, t = 0
\end{equation}

Where $\mathbf{u}$ is the velocity vector, $p$ is the pressure, $\nu$ is the kinematic viscosity, $\mathbf{f}$ are the external forces, $\mathbf{g}_D$ is the Dirichlet boundary condition, $\mathbf{g}_N$ is the Neumann boundary condition, $\mathbf{n}$ is the outward unit normal vector, $\mathbf{u}_0$ is the initial velocity and $\Omega$ is the domain of the fluid.

\subsection{Weak formulation}
The weak formulation of the Navier-Stokes equations is obtained by multiplying the equations by test functions and integrating over the domain $\Omega$.
To define the test functions, we introduce the following spaces:

\begin{equation}
    H^1_D(\Omega) = \{f \in H^1(\Omega) : f = 0 \text{ on } \Gamma_D\}
\end{equation}
\begin{equation}
    V = [H^1_D(\Omega)]^3
\end{equation}
\begin{equation}
    Q = L^2(\Omega)
\end{equation}

We may now multiply the equations by test functions ($\mathbf{v} \in V$ and $q \in Q$) and integrate over the domain $\Omega$. The weak formulation consists in finding $(\mathbf{u}, p) \in V \times Q$ such that:

\begin{equation}
    \begin{split}
        & \int_{\Omega} \frac{\partial \mathbf{u}}{\partial t} \mathbf{v} \, d\Omega + \int_{\Omega} \nu \nabla \mathbf{u} \cdot \nabla \mathbf{v} \, d\Omega + \int_{\Omega} ((\mathbf{u} \cdot \nabla) \mathbf{u}) \cdot \mathbf{v} \, d\Omega - \\
        & \int_{\Omega} p \nabla \cdot \mathbf{v} \, d\Omega = \int_{\Omega} \mathbf{f}_{\text{ext}} \cdot \mathbf{v} \, d\Omega + \int_{\partial \Omega} \mathbf{g}_N \cdot \mathbf{v} \, ds \quad \forall \mathbf{v} \in V
    \end{split}
\end{equation}
    
\begin{equation}
    \int_{\Omega} q \nabla \cdot \mathbf{u} d \Omega = 0 \forall q \in Q
\end{equation}

\subsection{Discretization}
We now discretize the weak formulation above using a semi-implicit time discretization and a finite element spatial discretization. The time derivative will be discretized using the finite difference method, therefore ($\mathbf{u}^{n+1} \cdot \nabla) \mathbf{u}^{n+1}$ and $\frac{\partial \mathbf{u}}{\partial t}$ will be discretized respectively as $(\mathbf{u}^n \cdot \nabla) \mathbf{u}^{n+1}$ and $\frac{\mathbf{u}^{n+1} - \mathbf{u}^n}{\Delta t}$, where $\Delta t$ is the time step.

After discretizing in time, space discretization is performed using the finite element method. Defining the space of polynomial functions of degree $r$ as $P_r$, we can introduce the finite element space:

\begin{equation}
    X_h^r = \{\mathbf{v}_h \in C^0(\bar\Omega) : \mathbf{v}_h|_K \in P_r, \forall K \in \mathcal{T}_h\}
\end{equation}

Where $\mathcal{T}_h$ is a triangulation of the domain $\Omega$. The aforementioned spaces $V$ and $Q$ can now be used to define the discrete spaces $V_h = (X_h^2)^3 \cap V$ and $Q_h = X_h^1 \cap Q$.
Given the exact solutions $\mathbf{u}$ and $p$ at time $t_n$, our goal is to find the approximate solutions $\mathbf{u}^n_h \in V_h$ and $p^n_h \in Q_h$ at time $t_n$, through the use of the finite element basis functions $\{\phi_i\}_{i=1}^{N_{\mathbf{u}}}$ and $\{\psi_j\}_{j=1}^{N_p}$, where $N_{\mathbf{u}}$ and $N_p$ are the number of degrees of freedom for the velocity and pressure respectively.

\begin{equation}
    \mathbf{u}^n_h = \sum_{i=1}^{N_{\mathbf{u}}} \mathbf{u}_i^n \phi_i(x) \quad \text{and} \quad p^n_h = \sum_{j=1}^{N_p} p_j^n \psi_j(x)
\end{equation}

This allows us to approximate the exact solutions $\mathbf{u}$ and $p$ as $\mathbf{u}^n_h$ and $p^n_h$ respectively.


\begin{equation}
    \begin{split}
        & \frac{1}{\Delta t} \int_{\Omega_h} \mathbf{u}^{n+1}_h \phi_i d \Omega + \int_{\Omega_h} \nabla \mathbf{u}^{n+1}_h \cdot \nabla \phi_i d \Omega + \int_{\Omega_h} ((\mathbf{u}^n_h \cdot \nabla) \mathbf{u}^{n+1}_h) \cdot \phi_i d \Omega - \\
        & \int_{\Omega_h} p^{n+1}_h \nabla \phi_i d \Omega = \int_{\Omega_h} \mathbf{f}_{\text{ext}} \cdot \phi_i d \Omega + \int_{\partial \Omega_h} \mathbf{g}_N(t_{n+1}) \cdot \phi_i d s + \\
        & \frac{1}{\Delta t} \int_{\Omega_h} \mathbf{u}^n_h \phi_i d \Omega \quad \forall i = 1, \dots, N_{\mathbf{u}}
    \end{split}
\end{equation}
\begin{equation}
    \int_{\Omega_h} \psi_j \nabla \mathbf{u}^{n+1}_h d \Omega = 0 \quad \forall j = 1, \dots, N_p
\end{equation}

\subsection{Linearization}
% I'm not sure if "linearization" is the correct term here.

We can now assemble the linear system of equations that we need to solve to obtain the approximate solutions $\mathbf{u}^n_h$ and $p^n_h$.
This problem can be written in the form $A \mathbf{x} = \mathbf{b}$, where $\mathbf{x}$ is the vector of unknowns, $\mathbf{b}$ is the right hand side vector and $A$ is the matrix of coefficients. $A$, $\mathbf{x}$ and $\mathbf{b}$ can be written as follows:

$$
\begin{matrix}
    A = \begin{bmatrix}
        F & B^T \\
        -B & 0
    \end{bmatrix} \quad
    \mathbf{x} = \begin{bmatrix}
        \mathbf{U}^{n+1}_h \\
        \mathbf{P}^{n+1}_h
    \end{bmatrix} \quad
    \mathbf{b} = \begin{bmatrix}
        \mathbf{G} \\
        \mathbf{0}
    \end{bmatrix}
\end{matrix}
$$

Where $F$ is the matrix of coefficients for the velocity, $B$ is the matrix of coefficients for the pressure and $\mathbf{G}$ is a known vector. In particular, $F$ is defined as follows:

\begin{equation}
    F = \frac{1}{\Delta t} M + A + C(\mathbf{U}^n) \quad \text{and} \quad B = -B^T
\end{equation}

Where $M$ is the mass matrix, $A$ is the stiffness matrix and $C(\mathbf{U}^n)$ is the convection matrix. The vector $\mathbf{U}^{n+1}$ contains the velocity unknowns $u_1, \dots, u_{N_{\mathbf{u}}}$, while $\mathbf{P}^{n+1}$ contains the pressure unknowns $p_1, \dots, p_{N_p}$.