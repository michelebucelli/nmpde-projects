The Navier-Stokes equations are a set of partial differential equations that describe the motion of viscous fluids. They are named after Claude-Louis Navier and George Gabriel Stokes, who independently developed the equations in the 19th century. The equations are used to model turbulence, which is important in many engineering applications.

This report will focus on the unsteady, incompressible Navier-Stokes equations in two and three dimensions. More specifically, we will consider the "flow past a cylinder" problem, which is a classic benchmark problem in computational fluid dynamics. The problem consists of a fluid flowing past a cylinder, which is placed in a channel. We will start with a brief description of the code structure, followed by a discussion on the mathematical background, with an analysis of the methods used and their convergence and stability properties. We will then present the results for the simulation of the fluid and discuss briefly the implemented preconditioners and their performance and parallel scaling.

All the implemented code is available at \url{https://github.com/FrancescoPesce/nmpde-project3-Allahakbari-Miotti-Pesce} and has been delivered making a pull request to \url{https://github.com/michelebucelli/nmpde-projects}. The code is written in C++ and uses the deal.II library, with the exception of some scripts in Python or Bash. The code was tested using Ubuntu 22.04 as an operating system and requires the set of mk modules for scientific computing. The code is parallelized using MPI and can be compiled using CMake.